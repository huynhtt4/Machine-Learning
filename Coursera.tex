% Options for packages loaded elsewhere
\PassOptionsToPackage{unicode}{hyperref}
\PassOptionsToPackage{hyphens}{url}
%
\documentclass[
]{article}
\usepackage{amsmath,amssymb}
\usepackage{lmodern}
\usepackage{iftex}
\ifPDFTeX
  \usepackage[T1]{fontenc}
  \usepackage[utf8]{inputenc}
  \usepackage{textcomp} % provide euro and other symbols
\else % if luatex or xetex
  \usepackage{unicode-math}
  \defaultfontfeatures{Scale=MatchLowercase}
  \defaultfontfeatures[\rmfamily]{Ligatures=TeX,Scale=1}
\fi
% Use upquote if available, for straight quotes in verbatim environments
\IfFileExists{upquote.sty}{\usepackage{upquote}}{}
\IfFileExists{microtype.sty}{% use microtype if available
  \usepackage[]{microtype}
  \UseMicrotypeSet[protrusion]{basicmath} % disable protrusion for tt fonts
}{}
\makeatletter
\@ifundefined{KOMAClassName}{% if non-KOMA class
  \IfFileExists{parskip.sty}{%
    \usepackage{parskip}
  }{% else
    \setlength{\parindent}{0pt}
    \setlength{\parskip}{6pt plus 2pt minus 1pt}}
}{% if KOMA class
  \KOMAoptions{parskip=half}}
\makeatother
\usepackage{xcolor}
\usepackage[margin=1in]{geometry}
\usepackage{color}
\usepackage{fancyvrb}
\newcommand{\VerbBar}{|}
\newcommand{\VERB}{\Verb[commandchars=\\\{\}]}
\DefineVerbatimEnvironment{Highlighting}{Verbatim}{commandchars=\\\{\}}
% Add ',fontsize=\small' for more characters per line
\usepackage{framed}
\definecolor{shadecolor}{RGB}{248,248,248}
\newenvironment{Shaded}{\begin{snugshade}}{\end{snugshade}}
\newcommand{\AlertTok}[1]{\textcolor[rgb]{0.94,0.16,0.16}{#1}}
\newcommand{\AnnotationTok}[1]{\textcolor[rgb]{0.56,0.35,0.01}{\textbf{\textit{#1}}}}
\newcommand{\AttributeTok}[1]{\textcolor[rgb]{0.77,0.63,0.00}{#1}}
\newcommand{\BaseNTok}[1]{\textcolor[rgb]{0.00,0.00,0.81}{#1}}
\newcommand{\BuiltInTok}[1]{#1}
\newcommand{\CharTok}[1]{\textcolor[rgb]{0.31,0.60,0.02}{#1}}
\newcommand{\CommentTok}[1]{\textcolor[rgb]{0.56,0.35,0.01}{\textit{#1}}}
\newcommand{\CommentVarTok}[1]{\textcolor[rgb]{0.56,0.35,0.01}{\textbf{\textit{#1}}}}
\newcommand{\ConstantTok}[1]{\textcolor[rgb]{0.00,0.00,0.00}{#1}}
\newcommand{\ControlFlowTok}[1]{\textcolor[rgb]{0.13,0.29,0.53}{\textbf{#1}}}
\newcommand{\DataTypeTok}[1]{\textcolor[rgb]{0.13,0.29,0.53}{#1}}
\newcommand{\DecValTok}[1]{\textcolor[rgb]{0.00,0.00,0.81}{#1}}
\newcommand{\DocumentationTok}[1]{\textcolor[rgb]{0.56,0.35,0.01}{\textbf{\textit{#1}}}}
\newcommand{\ErrorTok}[1]{\textcolor[rgb]{0.64,0.00,0.00}{\textbf{#1}}}
\newcommand{\ExtensionTok}[1]{#1}
\newcommand{\FloatTok}[1]{\textcolor[rgb]{0.00,0.00,0.81}{#1}}
\newcommand{\FunctionTok}[1]{\textcolor[rgb]{0.00,0.00,0.00}{#1}}
\newcommand{\ImportTok}[1]{#1}
\newcommand{\InformationTok}[1]{\textcolor[rgb]{0.56,0.35,0.01}{\textbf{\textit{#1}}}}
\newcommand{\KeywordTok}[1]{\textcolor[rgb]{0.13,0.29,0.53}{\textbf{#1}}}
\newcommand{\NormalTok}[1]{#1}
\newcommand{\OperatorTok}[1]{\textcolor[rgb]{0.81,0.36,0.00}{\textbf{#1}}}
\newcommand{\OtherTok}[1]{\textcolor[rgb]{0.56,0.35,0.01}{#1}}
\newcommand{\PreprocessorTok}[1]{\textcolor[rgb]{0.56,0.35,0.01}{\textit{#1}}}
\newcommand{\RegionMarkerTok}[1]{#1}
\newcommand{\SpecialCharTok}[1]{\textcolor[rgb]{0.00,0.00,0.00}{#1}}
\newcommand{\SpecialStringTok}[1]{\textcolor[rgb]{0.31,0.60,0.02}{#1}}
\newcommand{\StringTok}[1]{\textcolor[rgb]{0.31,0.60,0.02}{#1}}
\newcommand{\VariableTok}[1]{\textcolor[rgb]{0.00,0.00,0.00}{#1}}
\newcommand{\VerbatimStringTok}[1]{\textcolor[rgb]{0.31,0.60,0.02}{#1}}
\newcommand{\WarningTok}[1]{\textcolor[rgb]{0.56,0.35,0.01}{\textbf{\textit{#1}}}}
\usepackage{graphicx}
\makeatletter
\def\maxwidth{\ifdim\Gin@nat@width>\linewidth\linewidth\else\Gin@nat@width\fi}
\def\maxheight{\ifdim\Gin@nat@height>\textheight\textheight\else\Gin@nat@height\fi}
\makeatother
% Scale images if necessary, so that they will not overflow the page
% margins by default, and it is still possible to overwrite the defaults
% using explicit options in \includegraphics[width, height, ...]{}
\setkeys{Gin}{width=\maxwidth,height=\maxheight,keepaspectratio}
% Set default figure placement to htbp
\makeatletter
\def\fps@figure{htbp}
\makeatother
\setlength{\emergencystretch}{3em} % prevent overfull lines
\providecommand{\tightlist}{%
  \setlength{\itemsep}{0pt}\setlength{\parskip}{0pt}}
\setcounter{secnumdepth}{-\maxdimen} % remove section numbering
\ifLuaTeX
  \usepackage{selnolig}  % disable illegal ligatures
\fi
\IfFileExists{bookmark.sty}{\usepackage{bookmark}}{\usepackage{hyperref}}
\IfFileExists{xurl.sty}{\usepackage{xurl}}{} % add URL line breaks if available
\urlstyle{same} % disable monospaced font for URLs
\hypersetup{
  pdftitle={Practical Machine Learning},
  pdfauthor={HuynhTT},
  hidelinks,
  pdfcreator={LaTeX via pandoc}}

\title{Practical Machine Learning}
\author{HuynhTT}
\date{2022-10-03}

\begin{document}
\maketitle

\hypertarget{introduction}{%
\subsection{Introduction}\label{introduction}}

This is the final report for Coursera's Practical Machine Learning
course, which is part of John Hopkins' Data Science Specialization
track. In this project, we will use accelerometer data from 6
participants' belts, forearms, arms, and dumbbells to predict how they
performed the exercise. In the training set, this is the ``classe''
variable. On the training set, we train 2 models: Decision Tree and
Support Vector Machine using k-folds cross validation. The accuracy and
out of sample error rate are then calculated using a validation set
randomly selected from the training csv data. We select the best model
based on those numbers and use it to predict 20 cases using the test csv
set.

\hypertarget{data-download}{%
\subsection{Data download}\label{data-download}}

Using devices such as Jawbone Up, Nike FuelBand, and Fitbit, it is now
possible to collect a substantial amount of personal activity data at a
low cost. These devices are part of the quantified self movement, a
group of enthusiasts that frequently monitor themselves to better their
health, discover behavioral trends, or because they are tech nerds.
People frequently quantify how much of a certain task they perform, but
rarely quantify how well they perform it. The objective of this
experiment is to utilize data from accelerometers attached to the waist,
forearm, arm, and dumbbell of six individuals. They were instructed to
perform barbell lifts in five different ways, both correctly and badly.

This webpage contains additional information:
\url{http://groupware.les.inf.puc-rio.br/har}

Here are the training data for this undertaking:

\url{https://d396qusza40orc.cloudfront.net/predmachlearn/pml-training.csv}

Here are the available test data:

\url{https://d396qusza40orc.cloudfront.net/predmachlearn/pml-testing.csv}

\#\# Load Library and Dataloading

\begin{Shaded}
\begin{Highlighting}[]
\FunctionTok{set.seed}\NormalTok{(}\DecValTok{2022}\NormalTok{)}
\FunctionTok{library}\NormalTok{(ggplot2)}
\FunctionTok{library}\NormalTok{(caret)}
\end{Highlighting}
\end{Shaded}

\begin{verbatim}
## Loading required package: lattice
\end{verbatim}

\begin{Shaded}
\begin{Highlighting}[]
\FunctionTok{library}\NormalTok{(lattice)}
\FunctionTok{library}\NormalTok{(kernlab)}
\end{Highlighting}
\end{Shaded}

\begin{verbatim}
## 
## Attaching package: 'kernlab'
\end{verbatim}

\begin{verbatim}
## The following object is masked from 'package:ggplot2':
## 
##     alpha
\end{verbatim}

\begin{Shaded}
\begin{Highlighting}[]
\FunctionTok{library}\NormalTok{(rattle)}
\end{Highlighting}
\end{Shaded}

\begin{verbatim}
## Loading required package: tibble
\end{verbatim}

\begin{verbatim}
## Loading required package: bitops
\end{verbatim}

\begin{verbatim}
## Rattle: A free graphical interface for data science with R.
## Version 5.5.1 Copyright (c) 2006-2021 Togaware Pty Ltd.
## Type 'rattle()' to shake, rattle, and roll your data.
\end{verbatim}

\begin{Shaded}
\begin{Highlighting}[]
\FunctionTok{library}\NormalTok{(corrplot)}
\end{Highlighting}
\end{Shaded}

\begin{verbatim}
## corrplot 0.92 loaded
\end{verbatim}

\begin{Shaded}
\begin{Highlighting}[]
\FunctionTok{library}\NormalTok{(gbm)}
\end{Highlighting}
\end{Shaded}

\begin{verbatim}
## Loaded gbm 2.1.8.1
\end{verbatim}

\begin{Shaded}
\begin{Highlighting}[]
\FunctionTok{library}\NormalTok{(randomForest)}
\end{Highlighting}
\end{Shaded}

\begin{verbatim}
## randomForest 4.7-1.1
\end{verbatim}

\begin{verbatim}
## Type rfNews() to see new features/changes/bug fixes.
\end{verbatim}

\begin{verbatim}
## 
## Attaching package: 'randomForest'
\end{verbatim}

\begin{verbatim}
## The following object is masked from 'package:rattle':
## 
##     importance
\end{verbatim}

\begin{verbatim}
## The following object is masked from 'package:ggplot2':
## 
##     margin
\end{verbatim}

\begin{Shaded}
\begin{Highlighting}[]
\NormalTok{traincsv}\OtherTok{\textless{}{-}}\FunctionTok{read.csv}\NormalTok{(}\StringTok{"pml{-}training.csv"}\NormalTok{)}
\NormalTok{testcsv}\OtherTok{\textless{}{-}}\FunctionTok{read.csv}\NormalTok{(}\StringTok{"pml{-}testing.csv"}\NormalTok{)}
\end{Highlighting}
\end{Shaded}

\hypertarget{statistical-description-of-data}{%
\subsection{Statistical Description of
Data}\label{statistical-description-of-data}}

\begin{Shaded}
\begin{Highlighting}[]
\FunctionTok{dim}\NormalTok{(traincsv)}
\end{Highlighting}
\end{Shaded}

\begin{verbatim}
## [1] 19622   160
\end{verbatim}

\begin{Shaded}
\begin{Highlighting}[]
\FunctionTok{dim}\NormalTok{(testcsv)}
\end{Highlighting}
\end{Shaded}

\begin{verbatim}
## [1]  20 160
\end{verbatim}

The training set contains 160 variables and 19622 observations, however
the test set only contains 20 of each.

Before building the predictive model, we need to clean the data. Empty,
low-volatility columns are omitted because they have no predictive
value. In addition, the recommended data must also be processed.
Deleting or averaging is common practice. In this case, we will remove
them from the data set.

\begin{Shaded}
\begin{Highlighting}[]
\NormalTok{traincsv }\OtherTok{\textless{}{-}}\NormalTok{ traincsv[,}\FunctionTok{colMeans}\NormalTok{(}\FunctionTok{is.na}\NormalTok{(traincsv)) }\SpecialCharTok{\textless{}}\NormalTok{ .}\DecValTok{8}\NormalTok{]}
\NormalTok{traincsv }\OtherTok{\textless{}{-}}\NormalTok{ traincsv[,}\SpecialCharTok{{-}}\FunctionTok{c}\NormalTok{(}\DecValTok{1}\SpecialCharTok{:}\DecValTok{7}\NormalTok{)]}
\NormalTok{tt }\OtherTok{\textless{}{-}} \FunctionTok{nearZeroVar}\NormalTok{(traincsv)}
\NormalTok{traincsv }\OtherTok{\textless{}{-}}\NormalTok{ traincsv[,}\SpecialCharTok{{-}}\NormalTok{tt]}
\NormalTok{inTrain }\OtherTok{\textless{}{-}} \FunctionTok{createDataPartition}\NormalTok{(}\AttributeTok{y=}\NormalTok{traincsv}\SpecialCharTok{$}\NormalTok{classe, }\AttributeTok{p=}\FloatTok{0.7}\NormalTok{, }\AttributeTok{list=}\NormalTok{F)}
\NormalTok{train }\OtherTok{\textless{}{-}}\NormalTok{ traincsv[inTrain,]}
\NormalTok{valid }\OtherTok{\textless{}{-}}\NormalTok{ traincsv[}\SpecialCharTok{{-}}\NormalTok{inTrain,]}
\FunctionTok{dim}\NormalTok{(traincsv)}
\end{Highlighting}
\end{Shaded}

\begin{verbatim}
## [1] 19622    53
\end{verbatim}

Now that we have completed the process of deleting the superfluous
variables, we are going to divide the training set into a validation and
a sub training set. The testing set known as ``testcsv'' will not be
altered and will be utilized for the test cases of the final quiz.

Within the training set, there are a total of 53 variables and 19622
observations.

\hypertarget{building-the-models}{%
\subsection{Building The Models}\label{building-the-models}}

In this section, we will evaluate many well-known models, such as
Decision Trees and the Support Vector Machine (SVM). This is probably
more than we will need to test, but simply for the sake of comparison
and as a matter of good practice, we shall run them.

Establish a control system for the training that makes advantage of
5-fold cross validation.

\begin{Shaded}
\begin{Highlighting}[]
\NormalTok{cl }\OtherTok{\textless{}{-}} \FunctionTok{trainControl}\NormalTok{(}\AttributeTok{method=}\StringTok{"cv"}\NormalTok{, }\AttributeTok{number=}\DecValTok{5}\NormalTok{, }\AttributeTok{verboseIter=}\ConstantTok{FALSE}\NormalTok{)}
\end{Highlighting}
\end{Shaded}

\hypertarget{decision-tree}{%
\subsubsection{Decision Tree}\label{decision-tree}}

We will build a Decision Tree model

\begin{Shaded}
\begin{Highlighting}[]
\NormalTok{trees }\OtherTok{\textless{}{-}} \FunctionTok{train}\NormalTok{(classe}\SpecialCharTok{\textasciitilde{}}\NormalTok{., }\AttributeTok{data=}\NormalTok{train, }\AttributeTok{method=}\StringTok{"rpart"}\NormalTok{, }\AttributeTok{trControl =}\NormalTok{ cl, }\AttributeTok{tuneLength =} \DecValTok{5}\NormalTok{)}

\FunctionTok{fancyRpartPlot}\NormalTok{(trees}\SpecialCharTok{$}\NormalTok{finalModel)}
\end{Highlighting}
\end{Shaded}

\includegraphics{Coursera_files/figure-latex/unnamed-chunk-5-1.pdf}

\hypertarget{testing-for-decision-tree-model}{%
\subsubsection{Testing for Decision Tree
Model}\label{testing-for-decision-tree-model}}

\begin{Shaded}
\begin{Highlighting}[]
\NormalTok{p\_trees }\OtherTok{\textless{}{-}} \FunctionTok{predict}\NormalTok{(trees, valid)}
\NormalTok{ctrees }\OtherTok{\textless{}{-}} \FunctionTok{confusionMatrix}\NormalTok{(p\_trees, }\FunctionTok{factor}\NormalTok{(valid}\SpecialCharTok{$}\NormalTok{classe))}
\NormalTok{ctrees}
\end{Highlighting}
\end{Shaded}

\begin{verbatim}
## Confusion Matrix and Statistics
## 
##           Reference
## Prediction    A    B    C    D    E
##          A 1501  464  469  453  159
##          B   64  529  105  234  283
##          C  104  106  450  119  143
##          D    0   40    2  158   26
##          E    5    0    0    0  471
## 
## Overall Statistics
##                                           
##                Accuracy : 0.5283          
##                  95% CI : (0.5154, 0.5411)
##     No Information Rate : 0.2845          
##     P-Value [Acc > NIR] : < 2.2e-16       
##                                           
##                   Kappa : 0.3829          
##                                           
##  Mcnemar's Test P-Value : < 2.2e-16       
## 
## Statistics by Class:
## 
##                      Class: A Class: B Class: C Class: D Class: E
## Sensitivity            0.8967  0.46444  0.43860  0.16390  0.43530
## Specificity            0.6331  0.85546  0.90286  0.98618  0.99896
## Pos Pred Value         0.4928  0.43539  0.48807  0.69912  0.98950
## Neg Pred Value         0.9391  0.86938  0.88394  0.85757  0.88704
## Prevalence             0.2845  0.19354  0.17434  0.16381  0.18386
## Detection Rate         0.2551  0.08989  0.07647  0.02685  0.08003
## Detection Prevalence   0.5176  0.20646  0.15667  0.03840  0.08088
## Balanced Accuracy      0.7649  0.65995  0.67073  0.57504  0.71713
\end{verbatim}

\hypertarget{support-vector-machine}{%
\subsubsection{Support Vector Machine}\label{support-vector-machine}}

\begin{Shaded}
\begin{Highlighting}[]
\NormalTok{m\_svm }\OtherTok{\textless{}{-}} \FunctionTok{train}\NormalTok{(classe}\SpecialCharTok{\textasciitilde{}}\NormalTok{., }\AttributeTok{data=}\NormalTok{train, }\AttributeTok{method=}\StringTok{"svmLinear"}\NormalTok{, }\AttributeTok{trControl =}\NormalTok{ cl, }\AttributeTok{tuneLength =} \DecValTok{5}\NormalTok{, }\AttributeTok{verbose =} \ConstantTok{FALSE}\NormalTok{)}

\NormalTok{p\_svm }\OtherTok{\textless{}{-}} \FunctionTok{predict}\NormalTok{(m\_svm, valid)}
\NormalTok{csvm }\OtherTok{\textless{}{-}} \FunctionTok{confusionMatrix}\NormalTok{(p\_svm, }\FunctionTok{factor}\NormalTok{(valid}\SpecialCharTok{$}\NormalTok{classe))}
\NormalTok{csvm}
\end{Highlighting}
\end{Shaded}

\begin{verbatim}
## Confusion Matrix and Statistics
## 
##           Reference
## Prediction    A    B    C    D    E
##          A 1537  141   92   60   64
##          B   35  838   99   37  130
##          C   48   59  773  115   78
##          D   48   26   29  705   58
##          E    6   75   33   47  752
## 
## Overall Statistics
##                                          
##                Accuracy : 0.7825         
##                  95% CI : (0.7717, 0.793)
##     No Information Rate : 0.2845         
##     P-Value [Acc > NIR] : < 2.2e-16      
##                                          
##                   Kappa : 0.7235         
##                                          
##  Mcnemar's Test P-Value : < 2.2e-16      
## 
## Statistics by Class:
## 
##                      Class: A Class: B Class: C Class: D Class: E
## Sensitivity            0.9182   0.7357   0.7534   0.7313   0.6950
## Specificity            0.9152   0.9366   0.9383   0.9673   0.9665
## Pos Pred Value         0.8115   0.7357   0.7204   0.8141   0.8237
## Neg Pred Value         0.9657   0.9366   0.9474   0.9484   0.9336
## Prevalence             0.2845   0.1935   0.1743   0.1638   0.1839
## Detection Rate         0.2612   0.1424   0.1314   0.1198   0.1278
## Detection Prevalence   0.3218   0.1935   0.1823   0.1472   0.1551
## Balanced Accuracy      0.9167   0.8362   0.8458   0.8493   0.8307
\end{verbatim}

\hypertarget{result}{%
\subsection{Result:}\label{result}}

The accuracy of models is summarized:

\begin{Shaded}
\begin{Highlighting}[]
\FunctionTok{print}\NormalTok{(df)}
\end{Highlighting}
\end{Shaded}

\begin{verbatim}
##   models accuracy  error
## 1   TREE   0.5283 0.4717
## 2    SVM   0.7825 0.2175
\end{verbatim}

TREE: 0.5283

SVM: 0.7825

With an accuracy of 0.996 and an out-of-sample error rate of 0.004, the
Random Forest model is clearly the superior option. That is one model
that we have determined to be adequate enough to employ for our test
sets.

\hypertarget{prediction}{%
\subsection{Prediction}\label{prediction}}

Putting our test set through its paces using the SVM model to determine
the classe (5 levels) result for 20 different scenarios.

\begin{Shaded}
\begin{Highlighting}[]
\NormalTok{pd }\OtherTok{\textless{}{-}} \FunctionTok{predict}\NormalTok{(m\_svm, testcsv)}
\FunctionTok{print}\NormalTok{(pd)}
\end{Highlighting}
\end{Shaded}

\begin{verbatim}
##  [1] C A B C A E D D A A C A B A E E A B B B
## Levels: A B C D E
\end{verbatim}

\hypertarget{plotting}{%
\subsection{Plotting}\label{plotting}}

\hypertarget{correlation-matrix}{%
\subsubsection{Correlation matrix}\label{correlation-matrix}}

\begin{Shaded}
\begin{Highlighting}[]
\NormalTok{corrPlot }\OtherTok{\textless{}{-}} \FunctionTok{cor}\NormalTok{(train[, }\SpecialCharTok{{-}}\FunctionTok{length}\NormalTok{(}\FunctionTok{names}\NormalTok{(train))])}
\FunctionTok{corrplot}\NormalTok{(corrPlot, }\AttributeTok{method=}\StringTok{"color"}\NormalTok{)}
\end{Highlighting}
\end{Shaded}

\includegraphics{Coursera_files/figure-latex/unnamed-chunk-11-1.pdf}

\hypertarget{plotting-the-models}{%
\subsubsection{Plotting the models}\label{plotting-the-models}}

\begin{Shaded}
\begin{Highlighting}[]
\FunctionTok{plot}\NormalTok{(trees)}
\end{Highlighting}
\end{Shaded}

\includegraphics{Coursera_files/figure-latex/unnamed-chunk-12-1.pdf}

\end{document}
